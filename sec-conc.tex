\section{Conclusion}
\label{sec-conc} 
In this paper, we design and develop a novel ranking system \oursystem to support heterogenous scholarly entity ranking including article, author, affiliation and venue rankings. In order fit the various ranking scenarios, our system is equipped with four ranking metrics, namely relevance, importance, citation and time. Moreover, by combining with different entities profiling, \oursystem help users have a better knowledge of the academic information both from breadth and depth. Thus, we gave an answer about the question which entities are the most authoritative in the queried field of study.



%In this paper, we present a novel ranking system for scholarly data, which aims to process the big scholarly data, rank heterogenous entities and analyse scholarly data to help researchers retrieve influential and recent works to inspire their future works. We develop Time-Weighted PageRank for evaluating the importance of entities and assemble the article, venue and author for scholarly article ranking. Moreover, we construct a huge heterogenous academic data property graph based on its structure to manage and operate scholarly data efficiently. We have also designed visualization in the system to provide better service for researchers 