\documentclass[10pt,conference,letterpaper]{IEEEtran}
\special{papersize=8.5in,11in}
\usepackage{latexsym}
\usepackage{amsfonts}
\usepackage{amsmath}
\usepackage{amssymb}
\usepackage{color}
\usepackage{epsfig}
\usepackage{xspace}
\usepackage{graphicx,epstopdf}
\usepackage{stfloats}
%\usepackage{times}
\usepackage{subfigure}
\usepackage{cite}
\usepackage{balance}
\usepackage{amsmath, bm}
\usepackage[english]{babel}
\usepackage{array}

\pdfpagewidth=8.5in
\pdfpageheight=11in

\newcommand{\eat}[1]{}
\newcommand{\sttab}{\rule{0pt}{8pt}\\[-3ex]}
\newcounter{ccc}
\newcommand{\bcc}{\setcounter{ccc}{1}\theccc.}
\newcommand{\icc}{\addtocounter{ccc}{1}\theccc.}
\newcommand{\myhrule}{\rule[.5pt]{\hsize}{.5pt}}
\newcommand{\mat}[2]{{\begin{tabbing}\hspace{#1}\=\+\kill #2\end{tabbing}}}
\newcommand{\stitle}[1]{\vspace{0.5ex}\noindent{\bf #1}}
\newcommand{\etitle}[1]{\vspace{0.5ex}\noindent{\em \underline{#1}}}
\newcommand{\marked}[1]{\textcolor{red}{#1}}
\newcommand{\markedb}[1]{\textcolor{blue}{#1}}
\newcommand{\subsubtitle}[1]{\vspace{0.5ex}\noindent\underline{{\bf #1}}}

%\newcommand{\stab}{\rule{0pt}{8pt}\\[-1.6ex]}
%\newcommand{\sttab}{\rule{0pt}{8pt}\\[-2ex]}
\newcommand{\sstab}{\rule{0pt}{8pt}\\[-2ex]}
\newcommand{\bi}{\begin{itemize}}
\newcommand{\ei}{\end{itemize}}

\newcommand{\ie}{\emph{i.e.,}\xspace}
\newcommand{\eg}{\emph{e.g.,}\xspace}
\newcommand{\wrt}{\emph{w.r.t.}\xspace}
\newcommand{\aka}{\emph{a.k.a.}\xspace}
\newcommand{\kw}[1]{{\ensuremath {\mathsf{#1}}}\xspace}

\newcommand{\oursystem}{\kw{Athena}}
\newcommand{\sarank}{\kw{SARank}}

\begin{document}


\title{A Ranking and Profiling-based Scholarly Data Analysis System}
%\title{\oursystem: A Ranking-based Scholarly Analysis System}
\title{A Ranking-based Scholarly Analysis System}

% author names and affiliations
% use a multiple column layout for up to three different
% affiliations
\eat{
\author{\IEEEauthorblockN{Michael Shell}
\IEEEauthorblockA{School of Electrical and\\Computer Engineering\\
Georgia Institute of Technology\\
Atlanta, Georgia 30332--0250\\
Email: http://www.michaelshell.org/contact.html}
\and
\IEEEauthorblockN{Homer Simpson}
\IEEEauthorblockA{Twentieth Century Fox\\
Springfield, USA\\
Email: homer@thesimpsons.com}
\and
\IEEEauthorblockN{James Kirk\\ and Montgomery Scott}
\IEEEauthorblockA{Starfleet Academy\\
San Francisco, California 96678--2391\\
Telephone: (800) 555--1212\\
Fax: (888) 555--1212}}
}%%% eat original format author

\author{\IEEEauthorblockN{Junfeng Liu, Shuai Ma, Renjun Hu and Jinpeng Huai}
\IEEEauthorblockA{SKLSDE Lab, Beihang University, Beijing, China}
\IEEEauthorblockA{Beijing Advanced Innovation Center for Big Data and Brain Computing, Beijing, China}
\{liujunfeng, mashuai, hurenjun, huaijp\}@buaa.edu.cn}
%liujunfeng, mashuai, hurenjun, huaijp

\maketitle


\begin{abstract}
Scholarly analysis systems greatly enhance scientific knowledge discovery and propagation.
%
Recently, a number of scholarly analysis systems have been developed. However, these systems barely support ranking heterogenous scholarly entities under different metrics and, hence, fail to answer questions such as which authors/venues/affiliations are the most relevant/important in the queried field of study. In addition, existing systems adopt RDBMS or distributed file systems as their storage solutions, whereas the linked feature of scholarly data is largely ignored.
%
To this end, in this paper, we design and develop \oursystem, a novel scholarly analysis system. Our \oursystem supports heterogeneous scholarly entity rankings and is equipped with both traditional relevance- and citation-based and more recent importance-based ranking metrics. Moreover, \oursystem utilizes a popular graph database Neo4j as the storage solution. The linked structures are directly leveraged to facilitate the efficiency of scholarly data management.
%Based on the above, we have built an online service for scholarly analysis. 
We demonstrate the advantage of Neo4j-based storage compared with RDBMS, and two use cases in heterogeneous scholarly entity ranking and author profiling.
\end{abstract}
% Up to now, some systems have been developed with a tremendous amount of scholarly data and providing a range of functions to query keywords. However, these systems barely support different types ranking or various entities ranking (\itshape e.g.,\upshape venues, affiliation and authors) when researchers tend to get better knowledge about the keywords. In the paper, we design and develop a novel ranking system for scholarly data analysis that uses graph engine to efficiently manage the heterogeneous, evolving and dynamic natures of structured scholarly data and integrates different types ranking and various entities ranking based on SARank.  bring great convenience in querying scholarly data and discovering knowledge.


\IEEEpeerreviewmaketitle

\section{Introduction}
\label{sec-intro}


% Contributions: (1) heterogeneous entity ranking, (2) graph database for scholarly data management, (3) demonstration
%% Why heterogeneous entity ranking:
%% Why graph database:


% background
%Scholarly analysis systems greatly enhance scientific knowledge discovery and propagation.
The continuous advancements in science and engineering has contributed to an ever-expanding body of scientific literature.
As a result, it is becoming more and more challenging for people to follow the related research progress timely.
With this in the background, scholarly analysis systems are studied and developed to enhance scientific knowledge discovery and propagation. Basically, these systems enable to {\em rank} related scholarly entities (\eg articles, authors) given a query and, better, {\em profile} those retrieved entities.

%Scientific publications accelerate the dissemination of scientific discoveries all around the world. However, it is becoming more challenging to manage scientific advancement nowadays, due to the huge volume of scientific articles published.
%Consequently, ranking systems play a more important role for efficient scholarly data analysis than ever before.


% Search Engine | supporting entities | ranking metric | url
% Google Scholar | Article & Author | hybrid (full text, venue, author, how often and recently has been cited), mainly based on citation number
% Microsoft Academic | Article & Author | hybrid (how often and to which a publication is cited)
% Semantic Scholar | Article & Author | citation velocity
% CiteSeerX: | Article & Author | citation analysis: PageRank and coauthorship network
% Aminer: Article & author | Relevance, Year, #Citation
% AceMap: Article & author | authors, venues of key words, based on the numbers | https://acemap.info


%difference with existing systems
In the academic and industry, a number of academic search engines have been developed, \eg Microsoft Academic\footnote{https://academic.microsoft.com/}, Google Scholar\footnote{https://scholar.google.com/}, Semantic Scholar\footnote{https://www.semanticscholar.org/}, CiteSeerX~\cite{li2006citeseerx}, AMiner~\cite{tang2008arnetminer} and AceMap~\cite{tan2016acemap}.
%
Given a query, these systems rank scholarly articles according to publish time, relevance or other citation-based metrics. %For instance, Google Scholar ranks articles mainly based on the number of citation, while Semantic Scholar proposes to use the citation velocity, which is a weighted average number of article citations in the last three years. On the other hand, CiteSeerX exploits weighted PageRank on the citation networks to determine the ranks of articles~\cite{sun2007popularity}.
%Besides, they also allow to retrieve and rank authors given author names.
%
However, a majority of them do not support to rank other heterogenous entities, \eg authors, venues and affiliations, in a specific field of study, except AceMap and Microsoft Academic. Note that AceMap simply ranks those entities  based on the numbers of associated articles relevant to the query, and it is unclear how Microsoft Academic does. Moreover, the established ranking metrics are either too simple (\eg time and relevance) or biased to older articles (\eg citation-based). That is, they are insufficient to identify influential scholarly entities in an early stage.
%
To conclude, these systems fail to answer questions such as which authors/venues/affiliations are recently the most relevant/important in the queried field of study.


%Besides the ranking entities, existing systems also have problems to fulfill the query need with their ranking metrics.
%Researchers are more preferred to find the influential and recent articles in an early stage to inspire their future work.
%However, neither published time and relevance based nor tradition citation-based rankings (\eg based on citation numbers or PageRank scores) is able to identify those from massive scholarly articles. Note that the latter ranking metrics are well known to have bias to old articles.
%Recently, an importance-based scholarly article ranking model, namely \sarank, has proven effective for identifying recent and influential articles even when their citations have not been well observed yet~\cite{ma2018query}. It ranks scholarly articles by assembling the importance of involved entities such that the importance is a combination of prestige and popularity. %to capture the evolving nature of entities.


%\marked{the reason for using Neo4j}
Storage is another key factor for the success of scholarly analysis systems, due to the large volume (\eg the MAG data contains 126 and 529 million articles and citations, respectively~\cite{sinha2015overview}) and the complex relationships between heterogeneous entities.
%Traditional RDBMS are used by CiteSeerx, AMiner and Semantic Scholar to manage scholarly data. On the other hand, Acemap and Microsoft Academic exploit distributed file systems.
Existing systems have exploited RDBMS or distributed file systems as the storage solutions.
%
We observe that scholarly entities are inherently linked. The existing solutions all ignore such linked feature, and may be inferior in answering graph queries compared with structure-aware storage solutions~\cite{BigGraphSearch}. For instance, complex joins in RDBMS will become the bottleneck when processing queries such as {\em finding the top-k articles of an author}.
%To this end, we propose to utilize a popular graph database Neo4j to store and manage the large-scale, heterogeneous and linked scholarly data in \oursystem.

%may face challenges for efficient and high-concurrent query processing when the computing resources are limited

\eat{
\begin{table}
\caption{}
\centering
\begin{tabular*}{8cm}{llll}
\hline
 & Paper Info & Author's TopK Paper & TopK Cited Article Info\\
\hline
MySQL & 0.251 s  & 6.493 s & 35.190 s \\
\hline
Neo4j & 0.239 s  & 3.062 s & 24.332 s \\
\hline
\end{tabular*}
\end{table}
}

\begin{table}[t!]
\label{tab-function}
\begin{center}
\caption{Functions implemented in \oursystem and other systems}
\begin{scriptsize}
\begin{tabular}{|c|c c c c|c c c c|}
\hline
\multirow{2}{*}{\bf Systems}   &  \multicolumn{4}{c|}{\bf Query-dependent ranking}     & \multicolumn{4}{c|}{\bf Visual profiling}  \\
&  {\bf AR} & {\bf AU} & {\bf VE} & {\bf AF}  & {\bf AR} & {\bf AU} & {\bf VE} & {\bf AF} \\ \hline \hline
MS Academic & $\surd$ & $\surd$ & $\surd$ & $\surd$ & \marked{$\times$} & \marked{$\times$} & \marked{$\times$} & \marked{$\times$} \\
Google Scholar & $\surd$ & \marked{$\times$} & \marked{$\times$} & \marked{$\times$} & \marked{$\times$} & \marked{$\times$} & \marked{$\times$} & \marked{$\times$} \\
Semantic Scholar & $\surd$ & \marked{$\times$} & \marked{$\times$} & \marked{$\times$} & \marked{$\times$} & $\surd$ & \marked{$\times$} & \marked{$\times$} \\
CiteSeerX & $\surd$ & \marked{$\times$} & \marked{$\times$} & \marked{$\times$} &\marked{$\times$} & \marked{$\times$} & \marked{$\times$} & \marked{$\times$} \\
AMiner & $\surd$ & $\surd$ & \marked{$\times$} & \marked{$\times$} &  \marked{$\times$} & $\surd$ &  \marked{$\times$} &  \marked{$\times$}\\
AceMap & $\surd$ & $\surd$ & $\surd$ & $\surd$ & \marked{$\times$} & $\surd$ & $\surd$ & $\surd$ \\
\oursystem & $\surd$ & $\surd$ & $\surd$ & $\surd$ & $\surd$ & $\surd$ & $\surd$ & $\surd$ \\ \hline
\end{tabular} \\ \vspace{.5ex}
AR, AU, VE and AF stand for article, author, venue and affiliation, respectively.
\end{scriptsize}
\end{center}
\end{table}

%(in honor of her wisdom and justice)
\stitle{Contributions}.
To this end, in this paper, we design and develop \oursystem, a novel scholarly analysis system to facilitate deep understanding of scholarly data.

\noindent (1) \oursystem supports ranking four types of scholarly entities (\ie article, author, venue and affiliation) under various ranking metrics. Notably, we equip \oursystem with a recent importance-based ranking function that has proven effective for identifying recent influential articles even when their citations have not been well observed yet~\cite{ma2018query}.

%We develop a ranking model for various entities (\ie articles, author, affiliation and venue) by evaluating the importance of the entity in academic graph. And the importance captures the temporal nature of entities.

\noindent (2) \oursystem utilizes a popular graph database Neo4j to manage scholarly data. More specifically, we carefully design a Neo4j schema and represent the data as a huge property graph. We also incorporate Lucene index to enhance query processing.

%As entities in scholarly data are inherently linked, we construct a huge heterogenous academic data property graph based on graph engine and empirically verify the advantage in querying scholarly data compared with RDBMS.

\noindent (3) We have built an online service for \oursystem. In addition to rankings, we also implement heterogeneous entity profiling that presents some visual analysis for better understanding. The functions of \oursystem and other systems are summarized in Table~I. We demonstrate two use cases in scholarly entity ranking and profiling. We also demonstrate the advantage of storage with Neo4j compared with RDBMS.

%Our system implements a range of functions including: (i) article ranking by different ranking models given query keywords, (ii) heterogeneous entity rankings based on importance and (iii) detailed description and statistics of articles, authors, venues and affiliations.

%easily and efficiently search academic information by keywords; (ii) rank various scholarly entities (\itshape i.e., \upshape article, author, affiliation and venue) by different ranking type (\itshape i.e., \upshape SARank, relevance rank, citation and year); (ii) check home pages of affiliation, author, venue  \itshape etc.\upshape ; (iv) find detailed information about an article.

\stitle{Organization}.
The rest of this paper is organized as follows. Section \ref{sec-model} introduces the ranking model of \oursystem. The system overview is presented in Section \ref{sec-system}, followed by the demonstrations in Section~\ref{sec-demo} and conclusions in Section \ref{sec-conc}.
% which gives a brief description of our system,

\section{Ranking Model}
\label{sec-model}

\par
We first introduce Time-Weighted PageRank for evaluating the importance of entities, defined as a combination of the prestige and popularity, and then introduce entity ranking including author ranking, venue ranking, affiliation ranking and type ranking \itshape e.g., \upshape relevance ranking.


\subsection{Time-Weighted PageRank}
\par
PageRank is a typical method in scholarly articles ranking as we can easily make use of the reference between different articles. Due to the following problems, (1) xxx,
(2)xxx.
We introduce Time-Weighted PageRank(TWPageRank) by extending a time decay factor because of the impact of an article decay with time after peak time.
% introduce time weight pagerank

\par
We present TWPageRank that evaluate the prestige of nodes in a directed graph, in which each node attached with time information. And we use \itshape the impact weight \upshape to describe the relative weight from the edge sources to targets. Formally, the impact weight on a directed edge $(u,v), i.e.,$ an edge $u$ from $v$, is defined as:

\begin{equation}
\centering
\label{eq:edgeWeight}
w(u,v)=\left \{
\begin{array}{ccl}
1                        &    &{T_u   <   Peak_v} \\
e^{\sigma (T_u-Peak_v)}  &    &{T_u   \ge Peak_v} \\
\end{array} \right.
\end {equation}
where $T_u$ is the time of node $u$, $Peak_v$ is the peak time of node $v$ after which the impact weights of edges to $v$ decay with time, and $\sigma$ is a negative number controlling the decaying speed of the impacts. By default, we use years as its time granularity in Eq. (\ref{eq:edgeWeight}).

\par
Thus, the update rule of Time-weighted PageRank is

\begin{equation}
\centering
\label{eq:twPR}
PR(v)= d \sum_{(u,v)\in E} \frac{w(u,v)PR(u)}{W(u)} + \frac{1-d}{n}
\end{equation}
where $PR(u)$ and $PR(v)$ are the TWPageRank score of $u$ and $v$. And $E$ is a set of edges, $W(u)=\sum_v w(u,v)$ is the sum of the impact weights on all edges from $u$, $n$ is the number of nodes and $d$ is a damping parameter in (0, 1).
% what's TWpagerank


\subsection{Entity Ranking and Type Ranking}
\par
Scholarly entities(\itshape e.g., \upshape affiliation, venue, author and article) ranking is a problem of assessing the importance of nodes in a heterogeneous network. The importance is a combination of \itshape prestige \upshape and \itshape popularity \upshape to capture the evolving nature of entities. The prestige of scholarly entities is derived by applying TWPageRank on the citation graph $G$, and each type of entity is assigned the corresponding TWPageRank score as its prestige score $Prs$.

\par
To learn about the popularity of different scholarly entities, we first introduce the popularity of an article. The popularity of an article $v$ is the sum of all its citation freshness, \itshape i.e., \upshape the closeness to the current year:
\begin{equation}
\centering
\label{eq:pop}
Pop (v) = \sum_{(u,v) \in E^c} e^{\sigma(T_0 - T_u)}
\end{equation}
Here, $T_0$ is the current year, $T_u$ is the largest year among all articles, $\sigma$ is the negative decaying factor in Eq. (\ref{eq:edgeWeight}).

\par
Intuitively, prestige favors those with many citations soon after the publication of articles or associated articles of venues and authors, and popularity capture the temporal nature of entities.


\textbf{Affiliation Ranking.} We computes the importance of affiliations by their associated articles. As the importance of an affiliation evolves with time, we treat the affiliation importance in each year individually, and its importance is the sum of importance in all individual years.

\par
We construct an affiliation graph $G^a(V^a, E^a)$ using the citation information among affiliations, in which a node represents an affiliation in a specific year and a direct edge $(s, t)$  means that there exists articles of affiliation $s$ citing articles of affiliation $t$. Thus, the impact weights are defined as sums of impact weights from affiliation $s$ to affiliation $t$, \itshape i.e., \upshape
\begin{equation}
\centering
\label{eq:authorSumWeight}
w_a(s,t) = \sum_{u \in C(s), v \in C(t)} w(u,v)
\end{equation}
Here, $C(s)$ and $C(t)$ are the sets of articles of affiliation $s$ and affiliation $t$, and $w(u,v)$ is the impact weight of articles $u$ and $v$.
% define the impact weight between between author s and author t

\par
The prestige of an affiliation in a specific year ($Prs_a$) is computed using the impact weights Eq. (\ref{eq:authorSumWeight}) and the update rule in Eq. (\ref{eq:twPR}). The popularity of an affiliation in a specific year ($Pop_a$) is defined as the average popularity of its articles that is computed using Eq. (\ref{eq:pop}). Thus, the \itshape affiliation importance score \upshape ($Imp_a$) is defined as a combination of its prestige and popularity:
\begin{equation}
\centering
\label{eq:imp}
Imp_a(v) = Prs_a(v)^{\lambda} Pop_a(v)^{1-\lambda}
\end{equation}
Here, $\lambda \in [0, 1]$ is the importance factor, indicates the weighting about prestige and popularity.

\textbf{Venue Ranking.}
We computes the importance of venue using their associated articles which is similar with affiliation ranking. We treat the venue importance in each year, and construct a venue graph $G^v(V^v, E^v)$ using citation information among venues. And then we combine the prestige of a venue ($Prs_v$) and the popularity of venue ($Pop_v$) as the \itshape venue importance score \upshape ($Imp_v$).


\textbf{Author Ranking.}
We evaluate the importance of each author, and compute the average importance of the authors of an article as its \itshape author importance score. \upshape However, it is obvious that the author citation graph is too large to handle. Hence, we evaluate the prestige, popularity of the author by using the average prestige, popularity of all her/his published articles, respectively. Then, the author importance score ($Imp_{aut}$) is the combination of its prestige and popularity, similar to affiliation ranking.


\textbf{Article Ranking.}
If we are only to rank scholarly articles, the other type of entities such as venues and authors are closely involved. Hence, we assemble the importance of article, venue and author to produce the final scholarly articles ranking, illustrated in Fig. 1. Venue ranking and author ranking have presented in previous paragraphs. Next, introduce how to compute the importance of article using citation information.

\par
We first construct a citation graph $G^c(V^c, E^c)$ using citation information among articles. The prestige of articles ($Prs_c$) is derived by using TWPageRank in citation graph $G^c$. The popularity of an article ($Pop_c$)is the sum of all its freshness which has described in Eq. (\ref{eq:pop}). The importance of citation component ($Imp_c$) is a combination of its prestige and popularity in the citation graph by applying Eq. (\ref{eq:imp}).

\par
Thus, the static ranking score of an article $v$ is aggregated as follows:
\begin{equation}
\centering
\label{eq:assemble}
S(v) = \alpha Imp_v(v) + \beta Imp_{aut}(v) + (1 - \alpha - \beta)Imp_c(v)
\end{equation}
Here parameter $\alpha$, $\beta$ and $1- \alpha - \beta$  regularize the contributions of the venue, author and citation information.


\textbf{Relevance Ranking.} We have introduced affiliation, venue, author, article ranking in the former sections. These rankings are query independent and
aim to give a static ranking based on scholarly data only. However, it is vital to evaluate the similarity between the short query and sentence (\itshape i.e., \upshape title of the paper) when retrieve articles by keywords.

\par
Hence, we apply distributional semantic approach to represent words. Similarities between the term vectors indicates the corresponding semantic similarities \cite{corrado2013efficient}. The final ranking score of an article $v$ that relates to semantics of article title, defined as follow:
\begin{equation}
\centering
\label{eq:simscore}
F(v) = \theta S(v) + (1- \theta) \sum_j idf(Q_j) \frac {Q_j \cdot T} {\left \| Q_j \right \| \left \| T \right \|}
\end{equation}
Here, $S(v)$ is a static ranking score of article, $Q(j)$ is the j$th$ of query keywords vector which have removed stop words, $idf(Q_j)$ is inverse document frequency measures how much information the word $Q(j)$ provides, $T$ suggests we aggregate the title word vectors to their centroid and $\theta$ means the relevance factor.

\section{ Athena System}
%\section{System Framework}
\label{sec-system}

In this section, we introduce our \oursystem system.
As shown in Figure \ref{fig:framework}, \oursystem consists of three main components, \ie \emph{storage}, \emph{query engine} and {\em function modules}.
Below the storage component is a {\em query independent ranking} module that enriches the scholarly data with pre-computed query independent ranking scores,  there is another {\em query dependent ranking} module inside the query engine, and the two function modules support visual analyses for users.


%In this section, we introduce our \oursystem system. The system contains framework and schema, as shown in Figure \ref{fig:system}. The framework of \oursystem consists of three main components, \ie \emph{storage}, \emph{query engine} and {\em function modules} (Figure \ref{fig:framework}). The \emph{schema} is designed to model scholarly data as a property graph (Figure \ref{fig:schema}).

We next explain our system in detail.





\subsection{Schema Design} \label{subsec:schema}

Graph database Neo4j is adopted for storage in \oursystem. To do so, we need to design a schema that abstracts the entities and linked structures (\eg citation, authored-by).
% schema design rationaile
We follow two principles for schema design: (1) nodes for entities and relationships for linked structures, and (2) trading space for query efficiency if affordable and possible.

%reducing fine-grained relationship names while increase generic relationships qualified with property appropriately.

The schema is presented in Figure~\ref{fig:schema}, where the texts near nodes and relationships represent the properties of entities and linked structures, respectively.
It contains seven basic types of nodes including {\em Article}, {\em Author}, {\em Affiliation}, {\em Venue}, {\em FOS} (field of study), {\em ConIns} (conference instance in each year) and {\em Year}.
In addition, it further incorporates an artificial type of nodes, \ie~{\em AAA} representing article-author-affiliation tuples. Here we trade extra space for query efficiency, \ie an author and her/his affiliations can be retrieved in one query.
As another space-efficiency trade-off, we also use extra space to maintain certain properties of  {\em article} nodes: conference ID, journal ID and year.
%
Our schema also forms a total of nine types of relationships, one of which, \ie~{\em :VenueScore}, has a {\em score} property.



\begin{figure}
\centering
\subfigure[{\scriptsize Framework }]{\label{fig:framework}
\includegraphics[width=0.45\columnwidth]{systemFrame.pdf}}
%\hspace{3ex}
\subfigure[{\scriptsize Neo4j schema}]{\label{fig:schema}
\includegraphics[width=0.5\columnwidth]{neo4jSchema.pdf}}
\vspace{-1ex}
\caption{System design of \oursystem  }
\label{fig:system}
\vspace{-2ex}
\end{figure}


\subsection{Graph Storage} \label{subsec:storage}


Following the above schema, we maintain scholarly data as a huge property graph, \eg the one of MAG~\cite{sinha2015overview}  has more than 1.03 billion nodes and 1.93 billion relationships. %We further clarify our graph storage with the following.


First, based on the original scholarly data, the {\em query independent ranking} module pre-computes those query independent ranking scores: citation counts of articles, importance scores of articles, authors, venues and affiliations~\cite{ma2018query}. These scores are assigned as properties to the corresponding nodes.
%As iteratively accessing the linked entities is the most essential operation in the pre-computation, the graph storage is much more convenient and effective for such computations.
\marked{
Moreover, both citation counts and importance scores support incremental computation~\cite{ma2018query}, and they are easy to be dynamically maintained once new scholarly data arrives.%once the property graph gets updated.
}

Second, to facilitate query processing on the billion-scale property graph, Lucene index is utilized for initial entity lookups. Specifically, we create fulltext indices for article titles, author names, venue names and affiliation names. These enable to efficiently find articles, authors, venues and affiliations whose titles or names contain specific keywords. %Besides, we also create schema indices for the entire author and venue names to speed up initial entity lookups.


\subsection{Graph Query Engine } \label{subsec:qe}
%Utilizing Neo4j, \oursystem supports a variety of graph queries on the property graph to aid scholarly search. And the {\em query engine} is responsible for processing this set of queries. When a query is issued, the {\em Neo4j query engine} first translates it into a Neo4j Cypher query with proper parsing and semantic analyses. When applicable, the Cypher query also includes the entity IDs returned from the Lucene index. Relevance and relevant importance rankings given by the {\em query dependent ranking} module may also be included in the Cypher query if needed.Based on the final Cypher query, an optimized query plan is generated and processed on the property graph.

\marked{
Utilizing {\em Neo4j query engine}, \oursystem supports a variety of graph queries on the property graph to aid scholarly search, such as searching the author's top-k fields of study and searching the affiliation having a higher impact on which venues. When a query is issued, \oursystem first translates it into a Neo4j Cypher query.
When applicable, the Cypher query also includes the entity IDs returned from the Lucene index. Relevance and relevant importance rankings given by the {\em query dependent ranking} module may also be included in the Cypher query if needed.
Based on the final Cypher query, query plan is generated and processed on the property graph after proper parsing and semantic analyses.
}

%Find out which institutions have higher influence in which journals
%Athena turns user queries into database queries
%what kind of query


% neo4j query implement user query .
%

%Figure~\ref{fig:queryProcess}
%Table~\ref{tab-workflow}

Figure~\ref{fig:queryProcess} gives an example workflow of the query engine when a user wants to search the top scholarly articles about ``data mining"  ranked by relevant importance. The fulltext index is firstly used to get the related article IDs on ``data mining'' (lines 3--4). The {\em query dependent ranking} module then calculates the relevant importance scores of those related articles, and the top-k article IDs are further identified (lines 5--6). Based on the complete Cypher query, the {\em Neo4j query engine} finally generates the query plan, executes it on the property graph, and returns the results (lines 7--12).







\subsection{Function Modules}
Scholarly entity ranking and scholarly entity profiling are the two function modules that collect the ranked scholarly entities returned from the back-end, and present a visual analysis to users. More specifically, \oursystem utilizes RESTful APIs  and Echarts\footnote{ http://echarts.baidu.com} for the scholarly ranking and profiling. Further, \oursystem provides users with the APIs of the ranking and profiling functions.

\begin{figure}
\centering
\includegraphics[width=\columnwidth]{queryProcess.pdf}
\vspace{-3ex}
\caption{Example workflow of \oursystem query engine}
\label{fig:queryProcess}
\vspace{-2ex}
\end{figure}




% visualization do what ?
% restful API
% scenarios, Query and Ranking Scholarly Entity, Author Profiling.
%Based on the graph storage, scholarly data was managed and processed in the system back-end. \oursystem collects user querys, dispatch to query engine and the results are then presented by visualizer using RESTful API. We employ Echarts (http://echarts.baidu.com/) to display scholarly article analysis and author profiling, detailed demonstrations are accessed in next section.
% function implement in back-end, echarts js  using RESTful API. demonstrate xxx in the next section.


%With scholarly data management and processing in the system back-end, the visualizer of \oursystem collects user queries through user interfaces. The queries are to the query engine and the returned results are then presented by visualizer. We will demonstrate some scholarly analysis scenarios in the next Section.


%\begin{table}[t!]
%%\begin{center}
%\caption{An example workflow of the query engine}
%\vspace{-1ex}
%\label{tab-workflow}
%\begin{scriptsize}
%\begin{tabular}{ l}
%%\hline
%%{An example workflow of the query engine} \\
%\hline
%1. GraphDatabaseService graphDB = new GraphDatabaseFactory()... ; \\
%2.  \hspace{0ex}  try(Transaction tx = graphDB.beginTx()) \{ \\
%3.  \hspace{4ex} 	IndexHits $\langle$ Node$\rangle$ hits = db.index().forNodes(``fullTextIndex") \\
%4.  \hspace{8ex}   .query("title: graph AND title:database"); \\
%5.  \hspace{4ex} 	List$\langle$ ReleImpScore$\rangle$ listScore = calcReleImpo(hits, ``graph database"); \\
%6.  \hspace{4ex} 	List$\langle$ String$\rangle$  paIDs = getTopKIDs(listScore); \\
%7.  \hspace{4ex}    Result result = graphDB.execute(`` \\
%8.  \hspace{8ex}		WITH \{paIDs\} AS IDs UNWIND IDs AS perID \\
%9.  \hspace{8ex}		MATCH (p:Paper)-[:PaToPAA]-$>$(r)-[:PAAToAut]-(a:Author)\\
%10.	\hspace{7ex}		WHERE p.paID= perID \\
%11. \hspace{7ex}        WITH p.title,  COLLECT(a.auName), SIZE(()-[:PaRef]-$>$(p)) AS cite, r \\
%12.	\hspace{7ex}		RETURN  paID, title, authors, cite ... ");\\
%13.	\hspace{4ex}	    tx.success();\\
%14. \hspace{2ex}  \} \\
%\hline
%\end{tabular} \\ %\vspace{.5ex}
%\end{scriptsize}
%%\end{center}
%\end{table}


\section{System Demonstration}
\par The demonstration consists of three parts. (1) we demonstrate its ranking models and heterogenous entity rankings given the query keywords by \oursystem. (2) To further illustrate the effective ranking model based on importance, we take {\em time ranking} in SIGMOD as an example. (3) We also demonstrate author profiling for better knowledge of the author.


\stitle{Article Ranking and Keywords Profiling}. Fig. \ref{fig: search keywords} is an example of Search Page. In this page, we fulfill the query need with their ranking metrics and construct keywords profiling, try to answer which authors/venues/affiliations are the most authoritative in the queried field of study.

\par
We equip \oursystem with both article retrieval and other academic entity retrieval, such as author, affiliations, journal, conference series and conference instance. We present influential papers about the keywords and {\em importance} for default ranking metric. In order to fit various ranking scenarios, \oursystem supports different ranking metrics such as {\em relevance}, {\em importance}, {\em citation}, {\em year}. {\em Relevance} is more suitable for retrieving articles by keywords, because of capturing semantic information. While {\em importance} is more appropriate for ranking articles of affiliation and venue. We also demonstrate top-k prestige authors, influential affiliations, famous journals/conferences corresponding to ranking metrics and keywords.

\eat{
It has four major areas: (1) Area 1, the top of the picture, where users can search keywords, authors, affiliations, journal and conference (2) Area 2 presents influential papers about the keywords and {\em relevance} for default ranking metric, which is in the center of the picture. (3) At left of the picture is Area 3, users can specify different ranking metrics such as {\em relevance}, {\em importance}, {\em citation}, {\em year} to fit various ranking scenarios. (4) Top-k prestige authors, influential affiliations, famous journals/conferences corresponding to ranking metrics and keywords are shown in Area 4, which is at the right of the picture.
% keywords search and profiling.
}

\stitle{Ranking Instance} We rank the conference papers \eg SIGMOD following the metrics of {\em time ranking}. We only collect articles published earlier than 2016, so the top of {\em time ranking} is the maximum importance score in 2015. As shown in fig. \ref{fig:sigmod}, we put ``Spark SQL: Relational Data Processing in Spark" in second place, which has the most citations(653) in SIGMOD 2015 up to now. More generally, in our top 10, there are 3 articles that has the most citation in SIGMOD 2015.
% the description in ICDE 2018

\par
Although they share the same venue component in the same year, the author of the article has higher prestige and popularity, such as Matei Zaharia and Michael Armbrust. Besides, an article in VLDB cites the paper published in the same years that increases the prestige and popularity of the citation components. Thus, the paper possesses a higher importance score by assembling the component of citation, author and venue.

\stitle{Author profiling}. Fig. \ref{fig:hjwProfile} is an example of an Author Page, where contains author's basic information and author's detailed profiling. For basic information, users can check author's publications, related authors and author's affiliations. We also develop author's detailed profiling to have a knowledge of the author both from breadth and depth. Thus, we model the evolution of author's research interest, author's avatar with word cloud description, the statistics of publication, {\em etc}.
%author profiling

%\par
%\stitle{Affiliation profiling}. As shown in fig. \ref{}, we give an example of affiliation profiling. The layout of the Affiliation Page is similar with Search Page, users can discover publications using various ranking metrics and check statistics information, such as the importance author, relevant affiliation, famous journals/conferences.
%% affiliation profiling

%\par
%\stitle{Venue profiling} venue

\section{Conclusion}
\label{sec-conc} 
In this paper, we design and develop a novel ranking system \oursystem to support heterogenous scholarly entity ranking including article, author, affiliation and venue rankings. In order fit the various ranking scenarios, our system is equipped with four ranking metrics, namely relevance, importance, citation and time. Moreover, by combining with different entities profiling, \oursystem help users have a better knowledge of the academic information both from breadth and depth. Thus, we gave an answer about the question which entities are the most authoritative in the queried field of study.



%In this paper, we present a novel ranking system for scholarly data, which aims to process the big scholarly data, rank heterogenous entities and analyse scholarly data to help researchers retrieve influential and recent works to inspire their future works. We develop Time-Weighted PageRank for evaluating the importance of entities and assemble the article, venue and author for scholarly article ranking. Moreover, we construct a huge heterogenous academic data property graph based on its structure to manage and operate scholarly data efficiently. We have also designed visualization in the system to provide better service for researchers 




% trigger a \newpage just before the given reference
% number - used to balance the columns on the last page
% adjust value as needed - may need to be readjusted if
% the document is modified later
%\IEEEtriggeratref{8}
% The "triggered" command can be changed if desired:
%\IEEEtriggercmd{\enlargethispage{-5in}}

% references section

% can use a bibliography generated by BibTeX as a .bbl file
% BibTeX documentation can be easily obtained at:
% http://mirror.ctan.org/biblio/bibtex/contrib/doc/
% The IEEEtran BibTeX style support page is at:
% http://www.michaelshell.org/tex/ieeetran/bibtex/
%\bibliographystyle{IEEEtran}
% argument is your BibTeX string definitions and bibliography database(s)
%\bibliography{IEEEabrv,../bib/paper}
%
% <OR> manually copy in the resultant .bbl file
% set second argument of \begin to the number of references
% (used to reserve space for the reference number labels box)

% \begin{thebibliography}{1}

% \bibitem{IEEEhowto:kopka}
% H.~Kopka and P.~W. Daly, \emph{A Guide to \LaTeX}, 3rd~ed.\hskip 1em plus
%   0.5em minus 0.4em\relax Harlow, England: Addison-Wesley, 1999.

% \end{thebibliography}

\bibliographystyle{IEEEtran}
\bibliography{refs}

\end{document}


