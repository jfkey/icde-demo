\section{System Overview}
%\section{System Framework}
\label{sec-system}

\begin{figure}
\centering
\subfigure[{\scriptsize Framework}]{\label{fig:framework}
\includegraphics[width=0.4\columnwidth]{systemFrame.pdf}}
\hspace{3ex}
\subfigure[{\scriptsize Neo4j schema}]{\label{fig:schema}
\includegraphics[width=0.4\columnwidth]{neo4jSchema.pdf}}
\caption{System design of \oursystem}
\label{fig:system}
%\vspace{-2ex}
\end{figure}


Figure \ref{fig:framework} shows the framework of our \oursystem. It consists of two main components, \ie \emph{Storage} and \emph{Query Engine}.
Below the storage component is a {\em query-independent ranking} module that enriches the data with those pre-computed query-independent ranking metrics. Note that there is another {\em query-dependent ranking} module inside the query engine. Finally, two function modules on top present results to users based on the retrieved results. We next detail our system.

\subsection{Schema Design} \label{subsec:schema}

Graph database Neo4j is adopted for storage. As such, we need to design a schema that abstracts the entities and linked structures (\eg citation, authored-by).
% schema design rationaile
We follow two principles for schema design: (a) nodes for entities and relationships for linked structures, and (b) reducing fine-grained relationship names while increase generic relationships qualified with property appropriately.

The schema is presented in Fig.~\ref{fig:schema}.
It contains seven basic types of nodes including paper, author, affiliation, venue, field of study (FOS), conference instance (ConIns) and year. In addition, it further incorporates an artificial type of nodes, \ie PAA representing paper-author-affiliation tuples, to enhance query efficiency.
Finally, the property graph schema also forms a total of ten types of relationships to represent the linked structures between scholarly entities.



\subsection{Graph Storage} \label{subsec:storage}

%Thus, we model scholarly data as a huge heterogeneous graph, shown in Fig. \ref{fig:schema}, which contains more than one billion nodes and over two billion relationships.
% 126909021 paper  114698044 articles 529 million

Based on the above schema, we store the MAG data~\cite{sinha2015overview} as a huge property graph with \marked{more than one billion nodes and over two billion relationships (replacing one/two billion with concrete numbers).}
enables to store and manage our data as a graph.

\par
Neo4j with native graph storage ensures that scholarly data is stored efficiently and relationships close to each other. We have proved through experiments that index-free adjacency is more efficient and cheaper, shown in Section~\ref{sec-demo}. Moreover, we highlight our graph storage on the following two points.
% neo4j native graph storage.

\par
Firstly, given that our query independent ranking, a type of Time-Weighted PageRank based on graph, assesses the importance of nodes in the heterogeneous scholarly graph. Thus, it is much more convenient to compute the importance score when employ graph storage. Secondly,
we recompute the scholarly graph once scholarly data gets updated through incremental algorithm~\cite{ma2018query}. We can easily operate the affected areas of the scholarly graph on the graph storage. (weak? we do not implement in \oursystem) However, a huge volume index in RDBMS will cause significant performance bottleneck when update.
% benefits��PageRank algorithm. incremental computation. update index??? in neo4j and mysql. in stages

\par
We create index on property graph to deal with billion-scale scholarly data retrieval. Specifically, {\em Paper title} and {\em Affiliation name} are created fulltext index after stop word with IKAnalyzer to support Chinese. Other properties, such as {\em Author name} and {\em Paper ID}, are also created index for initial entity lookups. 
% lucene index. Why we need lucene index. index what. result.


\eat{
\marked{highlight graph storage, query engine detail}

\marked{relation to ranking model, when to do ranking}

\marked{incremental computation}


(2) property graph (billion-scale) -- lucene index
%(1) adopt neo4j
%(2)

Storage is a key factor for the success of scholarly analysis systems, due to the large volume of scholarly data (\eg \oursystem mainly uses the MAG data with 126 and 529 million articles and citations, respectively~\cite{sinha2015overview}) and the complex entities and relationships.
%
Traditional RDBMS are used by CiteSeerx, AMiner and Semantic Scholar to manage scholarly data. On the other hand, Acemap and Microsoft Academic exploit distributed file systems.
%
We observe that entities in scholarly data are inherently linked. The existing storage solutions all ignore such linked feature, and may face challenges for efficient and high-concurrent query processing when the computing resources are limited. For instance, complex join operations in RDBMS will become the bottleneck when processing queries such as {\em finding all articles of someone's co-authors}.
%
To this end, we propose to utilize a popular graph database Neo4j to store and manage the large-scale, heterogeneous and linked scholarly data in \oursystem.
}

%Scholarly data highly connected by reference relationship between articles and constructs a huge heterogeneous graph.  Take RDBMS as an example, complex joins and self-joins will incur obviously performance bottleneck when the scholarly dataset becomes more inter-related. A comparison of the performance of querying the cited articles of an article using RDBMS(Mysql) and graph database(Neo4j) is given in table 1. Furthermore, our heterogeneous entities ranking algorithm, a type of Time-Weighted PageRank bases on graph structure, assesses the importance of nodes in a heterogeneous graph. Thus, it utilizes a popular graph database Neo4j to store and manage heterogeneous scholarly data.
% scholarly data source, currently solution, why graph database.

%Our storage solution includes the property graph and transactions. The scholarly data is stored as a property graph with nodes and relationships and we use transactions to ensure the predictability of relationship-based queries. \marked{derivatives}
%  which possesses nodes and relationships


%Intuitively, a paper get published in a journal/conference by the author means new edges among paper, author and venue node. By employing ranking model as stated in section \ref{sec-model}, we derive affiliation, author, venue and article ranking score using incremental computation \cite{ma2018query}. And those score is described as a property in the graph schema.
% explain our schema



% design principles and schema.
% We take into consideration of the query ability of the graph schema and adopt specific time and space trade-offs. heterogeneous scholarly entity ranking ?

%In fact, we can apply any other ranking algorithms to rank scholarly entities in the graph schema.


\begin{figure*}[tp]
\centering
\includegraphics[width=\textwidth]{searchKeywords.pdf}
\caption{Search Keywords And Heterogeneous Entity Ranking}
\label{fig: search keywords}
\vspace{-3ex}
\end{figure*}


\subsection{Graph Query Engine} \label{subsec:qe}
\oursystem involves a variety of queries on property graph such as article retrieval, heterogeneous entity ranking and author topK fields of study. Query Engine is the component responsible for processing a common set of queries which consists of Query Dependent Ranking and Neo4j Query Engine. When users execute a query, cypher accesses the Neo4j Query Engine. After parser and semantic analysis, cypher is generated a query planner. It will hit Lucene index for initial entity lookups or directly traverse property graph according to the planner. After that, the result is sent to Query Dependent Ranking, where we compute the relevance or relevant importance score or rerank articles by other ranking metrics.

% query dependent ranking
% Neo4j Query Engine
% cypher
% query engine detail, cypher execute, cypher example. take system function for example

\par
Generally speaking, heterogeneous entity query and heterogeneous entity profiling are supported by \oursystem. An example procedure is provided in Table~\ref{tab-codeExample}. We present how Query Engine works when retrieves Topk articles about keywords ``graph database" by {\em Relevant Importance} ranking metrics.
%cypher example. give an example.


%1. GraphDatabaseService graphDB = new GraphDatabaseFactory().. .newGraphDatabase(); \\
%2.    try(Transaction tx = graphDB.beginTx()){ \\
%3.		IndexHits<Node> hits = db.index().forNodes("fulltext") \\
%								.query("titile: graph AND title:database"); \\
%4.		List<ReleImpScore> listScore = calcRelevantImportance(hits); \\
%5.		List<String> paIDs = getTopKIDs(listScore); \\
%6.		Result result = graphDB.execute(" \\
%			WITH {paIDs} AS coll UNWIND coll AS col \\
%			MATCH (p:Paper)-[:PaPAA]->(r)-[:PAAAth]-(a:Author)\\
%			WHERE p.paID= col  WITH ... \\
%			RETURN  title, paID, authors, authorsID;\\
%		");	\\
%7.		tx.success();\\
%	}



\eat{
(1) ranking computation \& incremental
(2) functionality based on query engine
(3) example

\oursystem supports a variety of queries on scholarly data such as keyword retrieval, subgraph search, heterogeneous entity ranking. Query engine is the component responsible for processing a common set of queries on graph database. It consists of the Lucene index, query optimization, Neo4j query engine and ranking algorithms.
% add heterogenous entity Ranking ?

\oursystem takes advantage of Lucene inverted index and \marked{employs Lucene index in property of paper titles, author names and distributed representations of words.
%
Query optimization aims to reduce cardinality of work in the progress to generate a new query plan, such as hitting index, reducing matching paths. Moreover, Neo4j query engine executes the query plan to perform efficient data retrieval.
Finally, search results are aggregated by SARank, relevance, citation, year, average and maximum functions.}
}


\subsection{Visualization}

% visualization do what ? 
% restful API 
% scenarios, Query and Ranking Scholarly Entity, Author Profiling. 
Based on the graph storage, scholarly data was managed and processed in the system back-end. \oursystem collects user querys, dispatch to query engine and the results are then presented by visualizer using RESTful API. We employ Echarts (http://echarts.baidu.com/) to display scholarly article analysis and author profiling, detailed demonstrations are accessed in next section.  
% function implement in back-end, echarts js  using RESTful API. demonstrate xxx in the next section. 

 
%With scholarly data management and processing in the system back-end, the visualizer of \oursystem collects user queries through user interfaces. The queries are to the query engine and the returned results are then presented by visualizer. We will demonstrate some scholarly analysis scenarios in the next Section.

\begin{figure}
\centering
\includegraphics[width=\columnwidth]{hjwAvatar.pdf}
\includegraphics[width=\columnwidth]{hjwInterest.pdf}
%\includegraphics[width=\columnwidth]{hjwPapers.pdf}
\caption{Author Profiling}
\label{fig:hjwProfile}
\vspace{-3ex}
\end{figure}


\begin{table}[t!]
%\begin{center}
\caption{Query TopK Articles by Relevant Importance }
\label{tab-codeExample}
\begin{scriptsize}
\begin{tabular}{ l}
\hline
{An Example procedure of Query TopK Articles by Relevant Importance } \\
\hline
1. GraphDatabaseService graphDB = new GraphDatabaseFactory()... ; \\
2.  \hspace{6ex}  try(Transaction tx = graphDB.beginTx()) \{ \\
3.  \hspace{12ex} 	IndexHits $\langle$ Node$\rangle$ hits = db.index().forNodes("fulltext") \\
    \hspace{20ex}   .query("titile: graph AND title:database"); \\
4.  \hspace{12ex} 	List$\langle$ ReleImpScore$\rangle$ listScore = calcRelevantImportance(hits); \\
5.  \hspace{12ex} 	List$\langle$ String$\rangle$  paIDs = getTopKIDs(listScore); \\
6.  \hspace{12ex}    Result result = graphDB.execute(`` \\
    \hspace{20ex}		WITH \{paIDs\} AS coll UNWIND coll AS col \\
    \hspace{20ex}		MATCH (p:Paper)-[:PaToPAA]-$>$(r)-[:PAAToAut]-(a:Author)\\
	\hspace{20ex}		WHERE p.paID= col  WITH p, r, a, ... \\
	\hspace{20ex}		RETURN  title, paID, authors, authorsID; ");\\
7.	\hspace{12ex}	tx.success();\\
    \hspace{6ex}  \} \\
\hline
\end{tabular} \\ %\vspace{.5ex}
\end{scriptsize}
%\end{center}
\end{table}
    