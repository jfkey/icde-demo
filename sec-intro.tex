\section{Introduction}
\label{sec-intro}


% Contributions: (1) heterogeneous entity ranking, (2) graph database for scholarly data management, (3) demonstration
%% Why heterogeneous entity ranking:
%% Why graph database:


% background
%Scholarly analysis systems greatly enhance scientific knowledge discovery and propagation.
The continuous advancements in science and engineering has contributed to an ever-expanding body of scientific literature.
As a result, it is becoming more and more challenging for people to follow the related research progress timely.
With this in the background, scholarly analysis systems are studied and developed to enhance scientific knowledge discovery and propagation. Basically, these systems enable to retrieve related scholarly entities (\eg articles, authors) given a query and, better, give an analysis of those retrieved.

%Scientific publications accelerate the dissemination of scientific discoveries all around the world. However, it is becoming more challenging to manage scientific advancement nowadays, due to the huge volume of scientific articles published.
%Consequently, ranking systems play a more important role for efficient scholarly data analysis than ever before.


% Search Engine | supporting entities | ranking metric | url
% Google Scholar | Article & Author | hybrid (full text, venue, author, how often and recently has been cited), mainly based on citation number
% Microsoft Academic | Article & Author | hybrid (how often and to which a publication is cited)
% Semantic Scholar | Article & Author | citation velocity
% CiteSeerX: | Article & Author | citation analysis: PageRank and coauthorship network
% Aminer: Article & author | Relevance, Year, #Citation
% AceMap: Article & author | authors, venues of key words, based on the numbers | https://acemap.info


%difference with existing systems
In the academic and industry, a number of academic search engines have been developed, \eg Microsoft Academic\footnote{https://academic.microsoft.com/}, Google Scholar\footnote{https://scholar.google.com/}, Semantic Scholar\footnote{https://www.semanticscholar.org/}, CiteSeerX~\cite{li2006citeseerx}, AMiner~\cite{tang2008arnetminer} and AceMap~\cite{tan2016acemap}.
%
Given query keywords, these systems rank scholarly articles in a variety of ways based on published time, relevance or by employing citation analysis. %For instance, Google Scholar ranks articles mainly based on the number of citation, while Semantic Scholar proposes to use the citation velocity, which is a weighted average number of article citations in the last three years. On the other hand, CiteSeerX exploits weighted PageRank on the citation networks to determine the ranks of articles~\cite{sun2007popularity}.
%
Besides, they also allow to retrieve and rank authors given author names.
%
However, none of them supports query-dependent rankings of other heterogenous entities, \eg authors, venues and affiliations, except AceMap~\cite{tan2016acemap} that is based on the numbers of associated articles relevant to the query keywords only.
%
As a result, these systems fail to answer questions such as which authors/venues/affiliations are the most authoritative in the queried field of study.


%Besides the ranking entities, existing systems also have problems to fulfill the query need with their ranking metrics.
%Researchers are more preferred to find the influential and recent articles in an early stage to inspire their future work.
%However, neither published time and relevance based nor tradition citation-based rankings (\eg based on citation numbers or PageRank scores) is able to identify those from massive scholarly articles. Note that the latter ranking metrics are well known to have bias to old articles.
%Recently, an importance-based scholarly article ranking model, namely \sarank, has proven effective for identifying recent and influential articles even when their citations have not been well observed yet~\cite{ma2018query}. It ranks scholarly articles by assembling the importance of involved entities such that the importance is a combination of prestige and popularity. %to capture the evolving nature of entities.


\marked{the reason for using Neo4j}
Storage is a key factor for the success of scholarly analysis systems, due to the large volume of scholarly data (\eg \oursystem mainly uses the MAG data with 126 and 529 million articles and citations, respectively~\cite{sinha2015overview}) and the complex entities and relationships.
%
Traditional RDBMS are used by CiteSeerx, AMiner and Semantic Scholar to manage scholarly data. On the other hand, Acemap and Microsoft Academic exploit distributed file systems.
%
We observe that entities in scholarly data are inherently linked. The existing storage solutions all ignore such linked feature, and may face challenges for efficient and high-concurrent query processing when the computing resources are limited. For instance, complex join operations in RDBMS will become the bottleneck when processing queries such as {\em finding all articles of someone's co-authors}.
%
To this end, we propose to utilize a popular graph database Neo4j to store and manage the large-scale, heterogeneous and linked scholarly data in \oursystem.
\marked{RDBMS vs. graph database moved to Introduction}

\begin{table}
\caption{}
\centering
\begin{tabular*}{8cm}{llll}
\hline
 & Paper Info & Author's TopK Paper & TopK Cited Article Info\\
\hline
MySQL & 0.251 s  & 6.493 s & 35.190 s \\
\hline
Neo4j & 0.239 s  & 3.062 s & 24.332 s \\
\hline
\end{tabular*}
\end{table}


\stitle{Contributions}.
To this end, in this paper, we design and develop a novel ranking system \oursystem (for her wisdom and justice) to support heterogeneous scholarly entity
ranking.

\noindent (1) We equip \oursystem with both tradition citation-based and the more advanced importance-based rankings. By evaluating the importance of entities, \oursystem supports heterogeneous entity ranking (\ie article, author, affiliation and venue rankings).

%We develop a ranking model for various entities (\ie articles, author, affiliation and venue) by evaluating the importance of the entity in academic graph. And the importance captures the temporal nature of entities.

\noindent (2) As entities in scholarly data are inherently linked, we construct a huge heterogenous academic data property graph based on graph engine and empirically verify the advantage in querying scholarly data compared with RDBMS.

\noindent (3) We build a prototype system for ranking and analyzing scholarly data. Our system implements a range of functions including: (i) article ranking by different ranking models given query keywords, (ii) heterogeneous entity rankings based on importance and (iii) detailed description and statistics of articles, authors, venues and affiliations.

%easily and efficiently search academic information by keywords; (ii) rank various scholarly entities (\itshape i.e., \upshape article, author, affiliation and venue) by different ranking type (\itshape i.e., \upshape SARank, relevance rank, citation and year); (ii) check home pages of affiliation, author, venue  \itshape etc.\upshape ; (iv) find detailed information about an article.


%Our system aims to provide better rankings for researchers that realizes a range of functions including: (1) easily and efficiently search academic information by keywords; (2) rank various scholarly entities (\itshape i.e., \upshape article, author, affiliation and venue) by different ranking type (\itshape i.e., \upshape SARank, relevance rank, citation and year); (3) check home pages of affiliation, author, venue  \itshape etc.\upshape ; (4) find detailed information about an article.


\stitle{Organization}.
The rest of this paper is organized as follows. Section \ref{sec-model} introduces the ranking models for articles, authors, venues and affiliations. We give a system overview in Section \ref{sec-system}, followed by the conclusions in Section \ref{sec-conc}.
% which gives a brief description of our system,
