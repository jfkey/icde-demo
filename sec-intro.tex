\section{Introduction}
\label{sec-intro}

%\begin{table*}[!t]
%\label{tab-function}
%\begin{center}
%\caption{Functions provided in \oursystem and existing scholarly search systems}
%%\vspace{-1ex}
%%\begin{scriptsize}
%\begin{tabular}{|c|c c c c|c c c c|c c c c|}
%\hline
%\multirow{2}{*}{Systems} &  \multicolumn{4}{c|}{Scholarly Search}   &  \multicolumn{4}{c|}{Article-Coupled Ranking}     & \multicolumn{4}{c|}{Visual Profiling}  \\
%&  {AR} & {AU} & {VE} & {AF} &   {AR} & {AU} & {VE} & {AF} & {AR} & {AU} & {VE} & {AF} \\
%%\hline
%\hline
%CiteSeerX & $\surd$  & $\surd$  & \marked{$\times$} & \marked{$\times$}
%& $\surd$ & \marked{$\times$} & \marked{$\times$} & \marked{$\times$} &\marked{$\times$} & \marked{$\times$} & \marked{$\times$} & \marked{$\times$} \\
%Google Scholar & $\surd$  & $\surd$  & \marked{$\times$} & $\surd$
% & $\surd$ & \marked{$\times$} & \marked{$\times$} & \marked{$\times$} & \marked{$\times$} & \marked{$\times$} & \marked{$\times$} & \marked{$\times$} \\
%Semantic Scholar & $\surd$ & $\surd$ & \marked{$\times$} & \marked{$\times$}
% & $\surd$ & \marked{$\times$} & \marked{$\times$} & \marked{$\times$} & \marked{$\times$} & $\surd$ & \marked{$\times$} &
%\marked{$\times$} \\
%AMiner & $\surd$ & $\surd$ & \marked{$\times$} & $\surd$
% & $\surd$ & $\surd$ & \marked{$\times$} & \marked{$\times$} & \marked{$\times$} & $\surd$ & \marked{$\times$} &   \marked{$\times$}\\
%MS Academic  & $\surd$ & $\surd$ & $\surd$ & $\surd$
%& $\surd$ & $\surd$ & $\surd$ & $\surd$ & \marked{$\times$} & \marked{$\times$} & \marked{$\times$} & \marked{$\times$} \\
%AceMap & $\surd$ & $\surd$ & $\surd$ & $\surd$
%& $\surd$ & $\surd$ & $\surd$ & \marked{$\times$} & \marked{$\times$} & $\surd$ & $\surd$ & $\surd$ \\
%\oursystem & $\surd$ & $\surd$ & $\surd$ & $\surd$
%& $\surd$ & $\surd$ &  $\surd$ & $\surd$ & $\surd$ & $\surd$ & $\surd$ & $\surd$ \\ \hline
%\end{tabular} \\
%\vspace{1ex}
%AR, AU, VE and AF stand for article, author, venue and affiliation,
%respectively.
%%\end{scriptsize}
%\vspace{-4ex}
%\end{center}
%\end{table*}


\begin{table*}[!t]
\label{tab-function}
\vspace{-3px}
\begin{center}
\caption{Functions provided by \oursystem and existing scholarly search systems}
\vspace{-3ex}
\begin{scriptsize}
\begin{tabular}{|c|c c c c|c c c c c|c c c c|c c c c|}
\hline
\multirow{2}{*}{Systems} &  \multicolumn{4}{c|}{Scholarly Search}
 &\multicolumn{5}{c|}{Ranking Metrics}   &  \multicolumn{4}{c|}{Article-Coupled Ranking}
 & \multicolumn{4}{c|}{Visual Profiling}  \\
&  {AR} & {AU} & {VE} & {AF} & {RE} & {IM} & {RE.IM}& {Time}& {Citation} & {AR} & {AU} & {VE} & {AF}   &  {AR} & {AU} & {VE} & {AF} \\
%\hline
\hline
CiteSeerX & $\surd$  & $\surd$  & \marked{$\times$} & \marked{$\times$}
& $\surd$ & \marked{$\times$} & \marked{$\times$} & $\surd$ & $\surd$
& \marked{$\times$} & \marked{$\times$} & \marked{$\times$} & \marked{$\times$}
& \marked{$\times$} & \marked{$\times$} & \marked{$\times$} & \marked{$\times$} \\
Google Scholar & $\surd$  & $\surd$  & \marked{$\times$} & $\surd$
 & $\surd$ & \marked{$\times$} & $\surd$ & $\surd$  & \marked{$\times$}
 & \marked{$\times$} & \marked{$\times$} & \marked{$\times$} & \marked{$\times$}
 & \marked{$\times$} & \marked{$\times$} & \marked{$\times$} & \marked{$\times$} \\
Semantic Scholar & $\surd$ & $\surd$ & \marked{$\times$} & \marked{$\times$}
 & $\surd$  & \marked{$\times$} & \marked{$\times$} & $\surd$ & $\surd$
 & \marked{$\times$} & \marked{$\times$} & \marked{$\times$} & \marked{$\times$}
 & \marked{$\times$} & $\surd$ & \marked{$\times$} & \marked{$\times$} \\
AMiner & $\surd$ & $\surd$ & \marked{$\times$} & $\surd$
 & $\surd$ & \marked{$\times$} & $\surd$ & $\surd$ & $\surd$
 & \marked{$\times$} & \marked{$\times$} & \marked{$\times$} & \marked{$\times$}
 & \marked{$\times$} & $\surd$ & \marked{$\times$} & \marked{$\times$}\\
AceMap & $\surd$ & $\surd$ & $\surd$ & $\surd$
& $\surd$  & \marked{$\times$} & \marked{$\times$} & \marked{$\times$} & \marked{$\times$}
& $\surd$ & \marked{$\times$} & $\surd$ & \marked{$\times$}
& \marked{$\times$} & $\surd$ & $\surd$ & $\surd$ \\
MS Academic  & $\surd$ & $\surd$ & $\surd$ & $\surd$
& $\surd$ & \marked{$\times$} & $\surd$ & $\surd$ & $\surd$
& $\surd$ & $\surd$  & \marked{$\times$} & $\surd$
& \marked{$\times$} & \marked{$\times$} & \marked{$\times$} & \marked{$\times$} \\
\oursystem & $\surd$ & $\surd$ & $\surd$ & $\surd$
& $\surd$ & $\surd$ &  $\surd$ & $\surd$
& $\surd$ & $\surd$ & $\surd$ & $\surd$ & $\surd$
& $\surd$ & $\surd$ & $\surd$ & $\surd$ \\ \hline
\end{tabular} \\
%\vspace{-3ex}
AR, AU, VE, AF, RE, IM, RE.IM stand for article, author, venue, affiliation, relevance, importance and relevant importance, respectively.
\end{scriptsize}
\vspace{-3ex}
\end{center}
\end{table*}

% reviewer 2
% The demo scenario, while interesting, is not very novel (esp. as seen from a comparison from Table I)��
% reviewer 3
% My reservation with respect to this submission regards novelty. As stated there are many academic search systems widely in use, and the contribution with respect to these existing systems, as detailed in Table 1, seems marginal.

% add ranking metrics. or remove same function.
% Article-Couple Ranking -> Entity-Couple Ranking, if AR equals \surd, means : in Article page, there support article-couple ranking for article, author, venue and affiliation. if AU equals \surd, means : in Author page, there support author-couple ranking for article, author, venue and affiliation. ...





% Contributions: (1) heterogeneous entity ranking, (2) graph database for scholarly data management, (3) demonstration
%% Why heterogeneous entity ranking:
%% Why graph database:


% background
%Scholarly analysis systems greatly enhance scientific knowledge discovery and propagation.
The continuous advancements in science and engineering have contributed to an ever-growing body of scientific literature.
To aid the deep understanding of scholarly data and to facilitate the research activities of scholars for scientific studies, a number of  scholarly search systems have been developed, which essentially provide searches and profiling of scholarly entities (articles, authors, venues and affiliations).

%Scientific publications accelerate the dissemination of scientific discoveries all around the world. However, it is becoming more challenging to manage scientific advancement nowadays, due to the huge volume of scientific articles published.
%Consequently, ranking systems play a more important role for efficient scholarly data analysis than ever before.


% Search Engine | supporting entities | ranking metric | url
% Google Scholar | Article & Author | hybrid (full text, venue, author, how often and recently has been cited), mainly based on citation number
% Microsoft Academic | Article & Author | hybrid (how often and to which a publication is cited)
% Semantic Scholar | Article & Author | citation velocity
% CiteSeerX: | Article & Author | citation analysis: PageRank and coauthorship network
% Aminer: Article & author | Relevance, Year, #Citation
% AceMap: Article & author | authors, venues of key words, based on the numbers | https://acemap.info


%difference with existing systems
Given a query of a specific research topic,  Google Scholar {\scriptsize (https://scholar.google.com)}, Semantic Scholar {\scriptsize (https://www.semanti-cscholar.org)} and CiteSeerX~\cite{li2006citeseerx} only rank scholarly articles, while AMiner~\cite{tang2008arnetminer} further provides author ranking. Besides, the supported ranking metrics in these systems are either simple (\eg time and relevance) or biased to older articles (\eg citation counts).
%
AceMap~\cite{tan2016acemap} and Microsoft Academic {\scriptsize (https://academic.microsoft.com)} do rank heterogeneous scholarly entities, \ie articles, authors, venues and affiliations. However, AceMap simply ranks those entities based on the numbers of associated articles with respect to the query, and Microsoft Academic does so by inferring query intent (based on the search log from Bing).
%
To conclude, these existing systems may face with limitations when users perform entity searches.

%\footnote{ https://scholar.google.com}
%\footnote{ https://www.semanticscholar.org}
%\footnote{https://academic.microsoft.com}

%Given a query, these systems rank scholarly articles according to publish time, relevance or other citation-based metrics. %For instance, Google Scholar
%However, a majority of them do not support ranking other heterogenous entities, \eg authors, venues and affiliations, in a specific field of study, expect that , and it is unclear how Microsoft Academic does \marked{focus on user intension Bing vs. we give a comprehensive recommendation based on relevance and importance}. Moreover, the established ranking metrics are either simple (\eg time and relevance) or biased to older articles (\eg citation-based). Hence, they are insufficient to identify influential scholarly entities in an early stage.
%


%ranks articles mainly based on the number of citation, while Semantic Scholar proposes to use the citation velocity, which is a weighted average number of article citations in the last three years. On the other hand, CiteSeerX exploits weighted PageRank on the citation networks to determine the ranks of articles~\cite{sun2007popularity}.
%Besides, they also allow to retrieve and rank authors given author names.


%\marked{the reason for using Neo4j}
Besides, storage is another key factor for scholarly search systems, due to the large volume of data and the complex relationships between entities, \eg Microsoft Academic Graph (MAG) contains 126/529 million articles/citations~\cite{sinha2015overview}.
Observe that scholarly entities are inherently linked. Existing systems mainly adopt RDBMS as their storage solutions. Hence they cannot fully exploit this linked feature, and become inferior when answering complex scholarly queries~\cite{BigGraphSearch}. For instance, joins in RDBMS become a bottleneck when {\em finding the top-k articles of an author}.

%To this end, we propose to utilize a popular graph database Neo4j to store and manage the large-scale, heterogeneous and linked scholarly data in \oursystem.

%may face challenges for efficient and high-concurrent query processing when the computing resources are limited
%compared with structure-aware storage solutions



%(in honor of her wisdom and justice)
\stitle{Contributions}.
In this study, we design and develop a scholarly search system \oursystem  to aid the deep understanding of scholarly data and to facilitate  the research activities of scholars  for scientific studies.

\noindent (1) \oursystem supports four types of entity searches with five ranking metrics. Besides the traditional ranking metrics (relevance, publish time and citation counts), we incorporate two advanced ranking metrics (importance and relevant importance), which has proven effective on ranking articles~\cite{ma2018query}.

\noindent (2) \oursystem also supports profiling scholarly entities, \eg author profiling with research interest evolution and affiliation profiling with author and field of study visualization.

\noindent  (3) \oursystem utilizes graph database Neo4j {\scriptsize (https://neo4j.com)} for storage. To do so, we carefully design a Neo4j schema to model scholarly data as a property graph, and incorporate Lucene index to speed up query processing.
%\footnote{https://neo4j.com}

\noindent (4) The advantages of \oursystem are demonstrated from four respects: scholarly search, profiling, graph storage and ranking quality.


%We develop a prototype system \oursystem, and demonstrate its advantages in the scholarly search, profiling, graph storage and ranking quality evaluation.

% We compare our Neo4j-based graph storage with the relational storage using MySQL and our ranking with PageRank and FutureRank~\cite{sayyadi2009futurerank}.

%We also provide online services via Web and API.

%Our system implements a range of functions including: (i) article ranking by different ranking models given query keywords, (ii) heterogeneous entity rankings based on importance and (iii) detailed description and statistics of articles, authors, venues and affiliations.

%easily and efficiently search academic information by keywords; (ii) rank various scholarly entities (\itshape i.e., \upshape article, author, affiliation and venue) by different ranking type (\itshape i.e., \upshape SARank, relevance rank, citation and year); (ii) check home pages of affiliation, author, venue  \itshape etc.\upshape ; (iv) find detailed information about an article.

%\stitle{Organization}.
The rest of this paper is organized as follows. Section \ref{sec-model} introduces scholarly search and ranking model. Our \oursystem system is presented in Section \ref{sec-system}, followed by demonstrations in Section~\ref{sec-demo} and conclusions in Section \ref{sec-conc}.





