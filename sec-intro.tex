\section{Introduction}
\label{sec-intro}


% Contributions: (1) heterogeneous entity ranking, (2) graph database for scholarly data management, (3) demonstration
%% Why heterogeneous entity ranking:
%% Why graph database:


% background
%Scholarly analysis systems greatly enhance scientific knowledge discovery and propagation.
The continuous advancements in science and engineering have contributed to an ever-expanding body of scientific literature.
As a result, it is becoming more and more challenging for people to follow the related research progress timely.
With this in the background, scholarly analysis systems are studied and developed to enhance scientific knowledge discovery and propagation. Basically, these systems enable to {\em rank} related scholarly entities (\eg articles, authors) given a query and, better, {\em profile} those retrieved entities.

%Scientific publications accelerate the dissemination of scientific discoveries all around the world. However, it is becoming more challenging to manage scientific advancement nowadays, due to the huge volume of scientific articles published.
%Consequently, ranking systems play a more important role for efficient scholarly data analysis than ever before.


% Search Engine | supporting entities | ranking metric | url
% Google Scholar | Article & Author | hybrid (full text, venue, author, how often and recently has been cited), mainly based on citation number
% Microsoft Academic | Article & Author | hybrid (how often and to which a publication is cited)
% Semantic Scholar | Article & Author | citation velocity
% CiteSeerX: | Article & Author | citation analysis: PageRank and coauthorship network
% Aminer: Article & author | Relevance, Year, #Citation
% AceMap: Article & author | authors, venues of key words, based on the numbers | https://acemap.info


%difference with existing systems
In the academic and industry, a number of scholarly analysis systems have been developed. Given a query of a specific research topic,  Google Scholar {\scriptsize (https://scholar.google.com/)}, Semantic Scholar {\scriptsize (https://www.semanticscholar.org/)} and CiteSeerX~\cite{li2006citeseerx} only ranks scholarly articles while AMiner~\cite{tang2008arnetminer} further gives rankings for authors. Also, the supported ranking metrics in these systems are either simple (\eg time and relevance) or biased to older articles (\eg citation-based).
%
On the other hand, Microsoft Academic {\scriptsize (https://academic.microsoft.com/)} and AceMap~\cite{tan2016acemap} do rank heterogenous entities, \ie articles, authors, venues and affiliations. However, AceMap simply ranks those entities based on the numbers of associated articles relevant to the query, and Microsoft Academic does by inferring query intent (based on the data from Bing).
%
To conclude, these systems may face challenges in ranking heterogeneous entities simultaneously with certain comprehensive metric.

%Given a query, these systems rank scholarly articles according to publish time, relevance or other citation-based metrics. %For instance, Google Scholar 
%However, a majority of them do not support ranking other heterogenous entities, \eg authors, venues and affiliations, in a specific field of study, expect that , and it is unclear how Microsoft Academic does \marked{focus on user intension Bing vs. we give a comprehensive recommendation based on relevance and importance}. Moreover, the established ranking metrics are either simple (\eg time and relevance) or biased to older articles (\eg citation-based). Hence, they are insufficient to identify influential scholarly entities in an early stage.
%


%ranks articles mainly based on the number of citation, while Semantic Scholar proposes to use the citation velocity, which is a weighted average number of article citations in the last three years. On the other hand, CiteSeerX exploits weighted PageRank on the citation networks to determine the ranks of articles~\cite{sun2007popularity}.
%Besides, they also allow to retrieve and rank authors given author names.


%\marked{the reason for using Neo4j}
Besides, storage is another key factor for the success of scholarly analysis systems, due to the large volume of data (\eg MAG~\cite{sinha2015overview} contains 126 and 529 million articles and citations, respectively) and the complex relationships between entities.
%Traditional RDBMS are used by CiteSeerx, AMiner and Semantic Scholar to manage scholarly data. On the other hand, Acemap and Microsoft Academic exploit distributed file systems.
Existing systems mainly exploit RDBMS as their storage solutions.
%
We observe that scholarly entities are inherently linked. Existing systems all ignore such linked feature, and may be inferior in answering complex scholarly queries~\cite{BigGraphSearch}. For instance, complex joins in RDBMS will become the bottleneck when {\em finding the top-k articles of an author}.

%To this end, we propose to utilize a popular graph database Neo4j to store and manage the large-scale, heterogeneous and linked scholarly data in \oursystem.

%may face challenges for efficient and high-concurrent query processing when the computing resources are limited
%compared with structure-aware storage solutions


\begin{table}[t!]
\label{tab-function}
\begin{center}
\caption{Functions implemented in \oursystem and other systems}
\vspace{-1ex}
\begin{scriptsize}
\begin{tabular}{|c|c c c c|c c c c|}
\hline
\multirow{2}{*}{\bf Systems}   &  \multicolumn{4}{c|}{\bf Field-dependent ranking}     & \multicolumn{4}{c|}{\bf Visual profiling}  \\
&  {\bf AR} & {\bf AU} & {\bf VE} & {\bf AF}  & {\bf AR} & {\bf AU} & {\bf VE} & {\bf AF} \\ \hline \hline
MS Academic & $\surd$ & $\surd$ & $\surd$ & $\surd$ & \marked{$\times$} & \marked{$\times$} & \marked{$\times$} & \marked{$\times$} \\
Google Scholar & $\surd$ & \marked{$\times$} & \marked{$\times$} & \marked{$\times$} & \marked{$\times$} & \marked{$\times$} & \marked{$\times$} & \marked{$\times$} \\
Semantic Scholar & $\surd$ & \marked{$\times$} & \marked{$\times$} & \marked{$\times$} & \marked{$\times$} & $\surd$ & \marked{$\times$} & \marked{$\times$} \\
CiteSeerX & $\surd$ & \marked{$\times$} & \marked{$\times$} & \marked{$\times$} &\marked{$\times$} & \marked{$\times$} & \marked{$\times$} & \marked{$\times$} \\
AMiner & $\surd$ & $\surd$ & \marked{$\times$} & \marked{$\times$} &  \marked{$\times$} & $\surd$ &  \marked{$\times$} &  \marked{$\times$}\\
AceMap & $\surd$ & $\surd$ & $\surd$ & $\surd$ & \marked{$\times$} & $\surd$ & $\surd$ & $\surd$ \\
\oursystem & $\surd$ & $\surd$ & $\surd$ & $\surd$ & $\surd$ & $\surd$ & $\surd$ & $\surd$ \\ \hline
\end{tabular} \\ \vspace{.5ex}
AR, AU, VE and AF stand for article, author, venue and affiliation, respectively.
\end{scriptsize}
\vspace{-4ex}
\end{center}
\end{table}

%(in honor of her wisdom and justice)
\stitle{Contributions}.
To this end, in this paper, we design and develop \oursystem, a novel scholarly analysis system to facilitate deep understanding of scholarly data. The functions of \oursystem and other systems are summarized in Table~I.

\noindent (1) \oursystem supports ranking four types of scholarly entities under five ranking metrics. Besides the traditional ones, we further equip \oursystem with a more advanced importance-ensemble metric that has proven effective for identifying important articles in an early stage~\cite{ma2018query}.

\noindent (2) \oursystem also supports profiling scholarly entities, \eg author profiling with research interest evolution and affiliation profiling with author and field of study visualization.

\noindent  (3) \oursystem utilizes graph database Neo4j {\scriptsize (https://neo4j.com)} for storage. To do this, we carefully design a Neo4j schema and represent the data as a huge property graph. We also incorporate Lucene index to enhance query efficiency.

\noindent (4) We have built a prototype system for \oursystem. We demonstrate two use cases in scholarly entity ranking and profiling and the advantage of storage with Neo4j compared with RDBMS. Our system also provides online service via API.

%Our system implements a range of functions including: (i) article ranking by different ranking models given query keywords, (ii) heterogeneous entity rankings based on importance and (iii) detailed description and statistics of articles, authors, venues and affiliations.

%easily and efficiently search academic information by keywords; (ii) rank various scholarly entities (\itshape i.e., \upshape article, author, affiliation and venue) by different ranking type (\itshape i.e., \upshape SARank, relevance rank, citation and year); (ii) check home pages of affiliation, author, venue  \itshape etc.\upshape ; (iv) find detailed information about an article.

\stitle{Organization}.
The rest of this paper is organized as follows. Section \ref{sec-model} introduces the ranking model of \oursystem. The system overview is presented in Section \ref{sec-system}, followed by the demonstration in Section~\ref{sec-demo} and conclusions in Section \ref{sec-conc}.
