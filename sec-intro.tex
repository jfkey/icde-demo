\section{Introduction}
\label{sec-intro}


% Contributions: (1) graph database for scholarly data management, (2) heterogeneous entity ranking, (3) demonstration
%% Why graph database:
%% Why heterogeneous entity ranking:

% background
Academic publications accelerate the dissemination of scientific discoveries all around the world. However, it is becoming more challenging to manage scientific advancement nowadays, due to the huge volume of scientific articles published. 
Consequently, ranking systems play a more important role for efficient scholarly data analysis than ever before. 

%Considering that the numbers of recorded articles and authors have reached 208 and 252 millions\footnote{https://academic.microsoft.com/}, respectively, an effective ranking system for scholarly data analysis is of great importance than ever before.

%Building a ranking system for scholarly data analysis is a critical and challenging task. due to the following reasons. First,  heterogeneous, evolving and dynamic nature of entities involved in scholarly articles. The large-scale of collection of academic publication.
%Scientific research plays a key role in promoting the development of society across the era. As a result of scientific
%research, scholarly articles accelerate the dissemination of scientific discoveries all around the world.
%The rise of large-scale collection of data on academic publications
%what we are doing now.
%retrieve and ranking is important .


% Search Engine | supporting entities | ranking metric | url
% Google Scholar | Article & Author | hybrid (full text, venue, author, how often and recently has been cited), mainly based on citation number
% Microsoft Academic | Article & Author | hybrid (how often and to which a publication is cited)
% Semantic Scholar | Article & Author | citation velocity
% CiteSeerX: | Article & Author | citation analysis: PageRank and coauthorship network
% Aminer: Article & author | Relevance, Year, #Citation
% AceMap: Article & author | authors, venues of key words, based on the numbers | https://acemap.info


In the academic and industry, a number of academic search engines have been developed, \eg Microsoft Academic\footnote{https://academic.microsoft.com/}, Google Scholar\footnote{https://scholar.google.com/}, Semantic Scholar\footnote{https://www.semanticscholar.org/}, CiteSeerX~\cite{li2006citeseerx}, AMiner~\cite{tang2008arnetminer} and AceMap~\cite{tan2016acemap}.
%
These systems enable to rank articles in a variety of  ways based on published time, relevance or by  employing citation analysis. For instance, Google Scholar ranks articles mainly based on the number of citation, while Semantic Scholar proposes to use the citation velocity, which is a weighted average number of article citations in the last three years. On the other hand, CiteSeerX exploits weighted PageRank on the citation networks to determine the ranks of articles~\cite{sun2007popularity}.
%
However, none of these systems supports ranking other heterogenous entities, \eg authors, venues and affiliations, given query words, except AceMap~\cite{tan2016acemap} that is based on the numbers of associated articles relevant to the query words only. As a result, these systems fail to answer questions such as which authors/venues/affiliations are the most authoritative in the queried field of study.


Besides the ranking entities, existing systems also have problems to fulfill the query need with their ranking metrics.
Researchers are more preferred to find the influential and recent articles in an early stage to inspire their future work.
However, neither published time and relevance based nor tradition citation-based rankings (\eg based on citation numbers or PageRank scores) is able to identify those from massive scholarly articles. Note that the latter ranking metrics are well known to have bias to old articles.
%
Recently, an importance-based scholarly article ranking model, namely \sarank, has proven effective for identifying recent and influential articles even when their citations have not been well observed yet~\cite{ma2018query}. It ranks scholarly articles by assembling the importance of involved entities such that the importance is a combination of prestige and popularity. %to capture the evolving nature of entities.



%~\cite{sinha2015overview}

%Actually Some academic search engines are prevalent over the past decade  e.g.,  Their common goal is to help researchers discover academic information and present academic data. However, there still remains several shortages in these systems, especially in ranking articles. For example, when rank queried articles some of them focus on citation of articles \cite{tang2008arnetminer,tan2016acemap}. While CiteSeerX evaluate articles employing weight PageRank and rank authors by coauthorship networks \cite{sun2007popularity,fiala2013citeseer}. They fails to capture the evolving nature of entities and take the involved entities (\itshape i.e,\upshape venues and authors) into consideration. Furthermore, some of them fail to support to query or rank affiliation, venue information which is also practicable to researchers \cite{tan2016acemap, googlescholar}.
% current academic search engine and its weakness


%Researchers prefer to retrieve influential and recent works in massive scholarly articles that will inspire their future works. In order to do this, we need to rank articles by making full use of the involved entities (\itshape i.e,\upshape articles,venues and authors) based on SARank \cite{ma2018query} and assessing the relevance of searching conditions and articles.
%why we need ranking system


%Generally speaking, a ranking is \itshape a function that assigns each item a numerical score. \upshape Scholarly articles involve with multi entities such as articles, authors, venues, affiliations and references which form a complex heterogeneous graph. Hence, scholarly articles ranking is essentially a problem of assessing the importance of nodes in a heterogeneous graph.


%Ranking such a huge heterogeneous graph and retrieving academic information remain some challenges. Firstly, high quality structured scholarly data is difficult to manage and operate that has exceeded 126 million articles and 114 million authors. Furthermore, academic data keeps increasing at around 5.7 million per year \cite{sinha2015overview}. Secondly, if we are only to rank one type of entities(\itshape i.e., \upshape scholarly articles), the other type of entities such as venues and authors are closely involved. Moreover, the impact of different types of entities on the ranking of scholarly articles differ from each other. Finally, the importance of entities change with time in a complex manner. Recently published articles are more likely to have a increasing impacts in next few years and those published many years ago tend to have decreasing impacts since researchers potentially interested in latest discoveries.

%what is ranking. challenges in ranking.



\stitle{Contributions}.
To this end, in this paper, we design and develop a novel ranking system, referred to as \oursystem, for supporting Heterogeneous Scholarly Entity
ranking. 

\noindent (1) We equip \oursystem with both tradition citation-based and the more advanced importance-based rankings. By evaluating the importance of entities, \oursystem supports heterogeneous entity ranking (\ie article, author, affiliation and venue rankings).

%We develop a ranking model for various entities (\ie articles, author, affiliation and venue) by evaluating the importance of the entity in academic graph. And the importance captures the temporal nature of entities.

\noindent (2) As entities in scholarly data are inherently linked, we construct a huge heterogenous academic data property graph based on graph engine and empirically verify the advantage in querying scholarly data compared with RDBMS.

\noindent (3) We build a prototype system for ranking and analyzing scholarly data. Our system implements a range of functions including: (i) article ranking by different ranking models given query keywords, (ii) heterogeneous entity rankings based on importance and (iii) detailed description and statistics of articles, authors, venues and affiliations.

%easily and efficiently search academic information by keywords; (ii) rank various scholarly entities (\itshape i.e., \upshape article, author, affiliation and venue) by different ranking type (\itshape i.e., \upshape SARank, relevance rank, citation and year); (ii) check home pages of affiliation, author, venue  \itshape etc.\upshape ; (iv) find detailed information about an article.


%Our system aims to provide better rankings for researchers that realizes a range of functions including: (1) easily and efficiently search academic information by keywords; (2) rank various scholarly entities (\itshape i.e., \upshape article, author, affiliation and venue) by different ranking type (\itshape i.e., \upshape SARank, relevance rank, citation and year); (3) check home pages of affiliation, author, venue  \itshape etc.\upshape ; (4) find detailed information about an article.


\stitle{Organization}.
The rest of this paper is organized as follows. Section \ref{sec-model} introduces the ranking models for articles, authors, venues and affiliations. We give a system overview in Section \ref{sec-system}, followed by the conclusions in Section \ref{sec-conc}.
% which gives a brief description of our system,