\section{Introduction}
\label{sec-intro}


% Contributions: (1) graph database for scholarly data management, (2) heterogeneous entity ranking, (3) demonstration 
%% Why graph database:
%% Why heterogeneous entity ranking: 

Academic publications accelerate the dissemination of scientific discoveries all around the world. Meanwhile, data on academic publications is so huge and keeps growing at explosive way. Building a ranking system for retrieving and analysing academic data is a critical and significant task.
% background

%Building a ranking system for scholarly data analysis is a critical and challenging task. due to the following reasons. First,  heterogeneous, evolving and dynamic nature of entities involved in scholarly articles. The large-scale of collection of academic publication.
%Scientific research plays a key role in promoting the development of society across the era. As a result of scientific
%research, scholarly articles accelerate the dissemination of scientific discoveries all around the world.
%The rise of large-scale collection of data on academic publications
%what we are doing now.
%retrieve and ranking is important .



Researchers prefer to retrieve influential and recent works in massive scholarly articles that will inspire their future works. In order to do this, we need to rank articles by making full use of the involved entities (\itshape i.e,\upshape articles,venues and authors) based on SARank \cite{ma2018query} and assessing the relevance of searching conditions and articles.
%why we need ranking system


Generally speaking, a ranking is \itshape a function that assigns each item a numerical score. \upshape Scholarly articles involve with multi entities such as articles, authors, venues, affiliations and references which form a complex heterogeneous graph. Hence, scholarly articles ranking is essentially a problem of assessing the importance of nodes in a heterogeneous graph.


Ranking such a huge heterogeneous graph and retrieving academic information remain some challenges. Firstly, high quality structured scholarly data is difficult to manage and operate that has exceeded 126 million articles and 114 million authors. Furthermore, academic data keeps increasing at around 5.7 million per year \cite{sinha2015overview}. Secondly, if we are only to rank one type of entities(\itshape i.e., \upshape scholarly articles), the other type of entities such as venues and authors are closely involved. Moreover, the impact of different types of entities on the ranking of scholarly articles differ from each other. Finally, the importance of entities change with time in a complex manner. Recently published articles are more likely to have a increasing impacts in next few years and those published many years ago tend to have decreasing impacts since researchers potentially interested in latest discoveries.

%what is ranking. challenges in ranking.

% Search Engine | supporting entities | ranking metric | url
% Google Scholar | Article & Author | hybrid (full text, venue, author, how often and recently has been cited), mainly based on citation number 
% Microsoft Academic | Article & Author | hybrid (how often and to which a publication is cited)
% Semantic Scholar | Article & Author | citation velocity
% CiteSeerX: | Article & Author | citation analysis: PageRank and coauthorship network
% Aminer: Article & author | Relevance, Year, #Citation  
% AceMap: Article & author | authors, venues of key words, based on the numbers | https://acemap.info

Actually academic search engine has drawn significant attentions from both academic and industry. Some academic search engines are prevalent over the past decade \itshape e.g., \upshape Google Scholar \cite{googlescholar}, Microsoft Academic \cite{sinha2015overview}, Semantic Scholar \cite{semantic}, CiteSeerX \cite{li2006citeseerx}, Aminer \cite{tang2008arnetminer} and AceMap \cite{tan2016acemap}. Their common goal is to help researchers discover academic information and present academic data. However, there still remains several shortages in these systems, especially in ranking articles. For example, when rank queried articles some of them focus on citation of articles \cite{tang2008arnetminer,tan2016acemap}. While CiteSeerX evaluate articles employing weight PageRank and rank authors by coauthorship networks \cite{sun2007popularity,fiala2013citeseer}. They fails to capture the evolving nature of entities and take the involved entities (\itshape i.e,\upshape venues and authors) into consideration. Furthermore, some of them fail to support to query or rank affiliation, venue information which is also practicable to researchers \cite{tan2016acemap, googlescholar}.
% current academic search engine and its weakness


\stitle{Contributions}.
To this end, we build a novel ranking system for scholarly data analysis which captures the evolving nature of entities and assembles the importance of involved entities. Our system aims to provide better rankings for researchers that realizes a range of functions including: (1) easily and efficiently search academic information by keywords; (2) rank various scholarly entities (\itshape i.e., \upshape article, author, affiliation and venue) by different ranking type (\itshape i.e., \upshape SARank, relevance rank, citation and year); (3) check home pages of affiliation, author, venue  \itshape etc.\upshape ; (4) find detailed information about an article.

\noindent (1) To describe the essence of articles reference, we construct a huge heterogenous academic data property graph based on graph engine that achieve great performance in querying academic data.

\noindent (2) We develop a ranking model for various entities (\itshape i.e., \upshape articles, author, affiliation and venue) by evaluating the importance of the entity in academic graph. And the importance captures the temporal nature of entities.

\noindent (3) We build a prototype system for ranking and analysing scholarly data that provide better services for researchers.


\stitle{Organization}.
The rest of our paper is organized as follows. Section 2 introduces the ranking model including author ranking, venue ranking, affiliation ranking and article ranking. We give a system overview in Section 3, which gives a brief description of our system. And the conclusion is shown in Section 4.
