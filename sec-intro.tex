\section{Introduction}
\label{sec-intro}


% Contributions: (1) heterogeneous entity ranking, (2) graph database for scholarly data management, (3) demonstration
%% Why heterogeneous entity ranking:
%% Why graph database:


% background
%Scholarly analysis systems greatly enhance scientific knowledge discovery and propagation.
The continuous advancements in science and engineering has contributed to an ever-expanding body of scientific literature.
As a result, it is becoming more and more challenging for people to follow the related research progress timely.
With this in the background, scholarly analysis systems are studied and developed to enhance scientific knowledge discovery and propagation. Basically, these systems enable to {\em rank} related scholarly entities (\eg articles, authors) given a query and, better, {\em profile} those retrieved entities.

%Scientific publications accelerate the dissemination of scientific discoveries all around the world. However, it is becoming more challenging to manage scientific advancement nowadays, due to the huge volume of scientific articles published.
%Consequently, ranking systems play a more important role for efficient scholarly data analysis than ever before.


% Search Engine | supporting entities | ranking metric | url
% Google Scholar | Article & Author | hybrid (full text, venue, author, how often and recently has been cited), mainly based on citation number
% Microsoft Academic | Article & Author | hybrid (how often and to which a publication is cited)
% Semantic Scholar | Article & Author | citation velocity
% CiteSeerX: | Article & Author | citation analysis: PageRank and coauthorship network
% Aminer: Article & author | Relevance, Year, #Citation
% AceMap: Article & author | authors, venues of key words, based on the numbers | https://acemap.info


%difference with existing systems
In the academic and industry, a number of academic search engines have been developed, \eg Microsoft Academic\footnote{https://academic.microsoft.com/}, Google Scholar\footnote{https://scholar.google.com/}, Semantic Scholar\footnote{https://www.semanticscholar.org/}, CiteSeerX~\cite{li2006citeseerx}, AMiner~\cite{tang2008arnetminer} and AceMap~\cite{tan2016acemap}.
%
Given a query, these systems rank scholarly articles according to publish time, relevance or other citation-based metrics. %For instance, Google Scholar ranks articles mainly based on the number of citation, while Semantic Scholar proposes to use the citation velocity, which is a weighted average number of article citations in the last three years. On the other hand, CiteSeerX exploits weighted PageRank on the citation networks to determine the ranks of articles~\cite{sun2007popularity}.
%Besides, they also allow to retrieve and rank authors given author names.
%
However, a majority of them do not support to rank other heterogenous entities, \eg authors, venues and affiliations, in a specific field of study, except AceMap and Microsoft Academic. Note that AceMap simply ranks those entities  based on the numbers of associated articles relevant to the query, and it is unclear how Microsoft Academic does. Moreover, the established ranking metrics are either too simple (\eg time and relevance) or biased to older articles (\eg citation-based). That is, they are insufficient to identify influential scholarly entities in an early stage.
%
To conclude, these systems fail to answer questions such as which authors/venues/affiliations are recently the most relevant/important in the queried field of study.


%Besides the ranking entities, existing systems also have problems to fulfill the query need with their ranking metrics.
%Researchers are more preferred to find the influential and recent articles in an early stage to inspire their future work.
%However, neither published time and relevance based nor tradition citation-based rankings (\eg based on citation numbers or PageRank scores) is able to identify those from massive scholarly articles. Note that the latter ranking metrics are well known to have bias to old articles.
%Recently, an importance-based scholarly article ranking model, namely \sarank, has proven effective for identifying recent and influential articles even when their citations have not been well observed yet~\cite{ma2018query}. It ranks scholarly articles by assembling the importance of involved entities such that the importance is a combination of prestige and popularity. %to capture the evolving nature of entities.


%\marked{the reason for using Neo4j}
Storage is another key factor for the success of scholarly analysis systems, due to the large volume (\eg the MAG data contains 126 and 529 million articles and citations, respectively~\cite{sinha2015overview}) and the complex relationships between heterogeneous entities.
%Traditional RDBMS are used by CiteSeerx, AMiner and Semantic Scholar to manage scholarly data. On the other hand, Acemap and Microsoft Academic exploit distributed file systems.
Existing systems have exploited RDBMS or distributed file systems as the storage solutions.
%
We observe that scholarly entities are inherently linked. The existing solutions all ignore such linked feature, and may be inferior in answering graph queries compared with structure-aware storage solutions~\cite{BigGraphSearch}. For instance, complex joins in RDBMS will become the bottleneck when processing queries such as {\em finding the top-k articles of an author}.
%To this end, we propose to utilize a popular graph database Neo4j to store and manage the large-scale, heterogeneous and linked scholarly data in \oursystem.

%may face challenges for efficient and high-concurrent query processing when the computing resources are limited

\eat{
\begin{table}
\caption{}
\centering
\begin{tabular*}{8cm}{llll}
\hline
 & Paper Info & Author's TopK Paper & TopK Cited Article Info\\
\hline
MySQL & 0.251 s  & 6.493 s & 35.190 s \\
\hline
Neo4j & 0.239 s  & 3.062 s & 24.332 s \\
\hline
\end{tabular*}
\end{table}
}

\begin{table}[t!]
\label{tab-function}
\begin{center}
\caption{Functions implemented in \oursystem and other systems}
\vspace{-2ex}
\begin{scriptsize}
\begin{tabular}{|c|c c c c|c c c c|}
\hline
\multirow{2}{*}{\bf Systems}   &  \multicolumn{4}{c|}{\bf Query-dependent ranking}     & \multicolumn{4}{c|}{\bf Visual profiling}  \\
&  {\bf AR} & {\bf AU} & {\bf VE} & {\bf AF}  & {\bf AR} & {\bf AU} & {\bf VE} & {\bf AF} \\ \hline \hline
MS Academic & $\surd$ & $\surd$ & $\surd$ & $\surd$ & \marked{$\times$} & \marked{$\times$} & \marked{$\times$} & \marked{$\times$} \\
Google Scholar & $\surd$ & \marked{$\times$} & \marked{$\times$} & \marked{$\times$} & \marked{$\times$} & \marked{$\times$} & \marked{$\times$} & \marked{$\times$} \\
Semantic Scholar & $\surd$ & \marked{$\times$} & \marked{$\times$} & \marked{$\times$} & \marked{$\times$} & $\surd$ & \marked{$\times$} & \marked{$\times$} \\
CiteSeerX & $\surd$ & \marked{$\times$} & \marked{$\times$} & \marked{$\times$} &\marked{$\times$} & \marked{$\times$} & \marked{$\times$} & \marked{$\times$} \\
AMiner & $\surd$ & $\surd$ & \marked{$\times$} & \marked{$\times$} &  \marked{$\times$} & $\surd$ &  \marked{$\times$} &  \marked{$\times$}\\
AceMap & $\surd$ & $\surd$ & $\surd$ & $\surd$ & \marked{$\times$} & $\surd$ & $\surd$ & $\surd$ \\
\oursystem & $\surd$ & $\surd$ & $\surd$ & $\surd$ & $\surd$ & $\surd$ & $\surd$ & $\surd$ \\ \hline
\end{tabular} \\ \vspace{.5ex}
AR, AU, VE and AF stand for article, author, venue and affiliation, respectively.
\end{scriptsize}
\vspace{-3ex}
\end{center}
\end{table}

%(in honor of her wisdom and justice)
\stitle{Contributions}.
To this end, in this paper, we design and develop \oursystem, a novel scholarly analysis system to facilitate deep understanding of scholarly data.

\noindent (1) \oursystem supports ranking four types of scholarly entities (\ie article, author, venue and affiliation) under various ranking metrics. Notably, we equip \oursystem with a recent importance-based ranking function that has proven effective for identifying recent influential articles even when their citations have not been well observed yet~\cite{ma2018query}.

%We develop a ranking model for various entities (\ie articles, author, affiliation and venue) by evaluating the importance of the entity in academic graph. And the importance captures the temporal nature of entities.

\noindent (2) \oursystem utilizes a popular graph database Neo4j\footnote{https://neo4j.com} to manage scholarly data. More specifically, we carefully design a Neo4j schema and represent the data as a huge property graph. We also incorporate Lucene index to enhance query processing.

%As entities in scholarly data are inherently linked, we construct a huge heterogenous academic data property graph based on graph engine and empirically verify the advantage in querying scholarly data compared with RDBMS.

\noindent (3) We have built an online service for \oursystem. In addition to rankings, we also implement heterogeneous entity profiling that presents some visual analysis for better understanding. The functions of \oursystem and other systems are summarized in Table~I. We demonstrate two use cases in scholarly entity ranking and profiling. We also demonstrate the advantage of storage with Neo4j compared with RDBMS.

%Our system implements a range of functions including: (i) article ranking by different ranking models given query keywords, (ii) heterogeneous entity rankings based on importance and (iii) detailed description and statistics of articles, authors, venues and affiliations.

%easily and efficiently search academic information by keywords; (ii) rank various scholarly entities (\itshape i.e., \upshape article, author, affiliation and venue) by different ranking type (\itshape i.e., \upshape SARank, relevance rank, citation and year); (ii) check home pages of affiliation, author, venue  \itshape etc.\upshape ; (iv) find detailed information about an article.

\stitle{Organization}.
The rest of this paper is organized as follows. Section \ref{sec-model} introduces the ranking model of \oursystem. The system overview is presented in Section \ref{sec-system}, followed by the demonstrations in Section~\ref{sec-demo} and conclusions in Section \ref{sec-conc}.
% which gives a brief description of our system,
